\documentclass[10pt,a4paper]{article}
\usepackage[utf8x]{inputenc}
\usepackage{ucs}
\usepackage{amsmath}
\usepackage{amsfonts}
\usepackage{amssymb}
\usepackage{graphicx}
\usepackage{hyperref}

\title{FFPB App collection demo}
\newcommand{\krombel}{Krombel's Dash }
\newcommand{\nodes}{nodes.json }
\newcommand{\wifiMapper}{WiFi Mapper Extension }
\begin{document}
\maketitle

\begin{abstract}
 The goal of this project is to develop multiple apps that can interact with each other, either via Intents or data exchange (content providers). 
\end{abstract}

%\tableofcontents

\section{Goals}
\begin{enumerate}
 \item Two different apps, that use the same Account in Android's Sync Framework.
 \item A common navigation drawer for both apps
\end{enumerate} 

\section{\krombel} Shows data from \url{http://dashing.krombel.de/stats}
\section{\nodes}
Fetches data from \url{http://map.paderborn.freifunk.net/nodes.json}. Displays the status off all nodes.

Steps:
\begin{enumerate}
 \item Display all nodes
 \item ``Watch my nodes''
 \item Integrate \wifiMapper
\end{enumerate}


\section{NavDrawer}
 Used to implement the common navigation drawer. The drawer's content is created dynamically and depends on the other apps' availability, i.e.
\begin{itemize}
 \item \krombel navigation is only shown, if \krombel is installed
 \item \nodes navigation is only shown, if \nodes is installed
\end{itemize}

\section{\wifiMapper}
Not a stand-alone app. This app is used in the \nodes app to collect Wifi Stats and send them to a server. 

\end{document}